% !TeX spellcheck = ru_RU
% !TEX root = vkr.tex

\section*{Введение}
\thispagestyle{withCompileDate}

Операции линейной алгебры составляют вычислительное ядро многих современных приложений: от машинного обучения и компьютерного зрения до научных расчётов и финансового моделирования. Среди них операция умножения матриц общего вида (GEMM, General Matrix Multiply) является наиболее зависимой от производительности, поскольку её эффективная реализация определяет скорость работы алгоритмов свёрточных нейронных сетей, итерационных решателей систем линейных уравнений и многих других методов.

Для ускорения таких вычислений активно используются графические процессоры (GPU) и специализированные ускорители, обладающие массивно-параллельной архитектурой. Однако разработка эффективного программного обеспечения для GPU осложняется разнообразием аппаратных платформ: различные производители (NVIDIA, AMD, Intel) предлагают собственные архитектуры и программные интерфейсы. Наиболее распространённый API для GPU вычислений --- CUDA~\cite{nickolls2008cuda} --- является проприетарной технологией NVIDIA и не поддерживается на устройствах других производителей.

Альтернативой проприетарным решениям служит открытый стандарт OpenCL~\cite{opencl_spec} (Open Computing Language) --- кросс-платформенный фреймворк для параллельного программирования, позволяющий разрабатывать приложения, исполняемые на различных вычислительных устройствах независимо от производителя. OpenCL обеспечивает переносимость кода и абстрагирование от специфики аппаратной архитектуры, что делает его резонным выбором для создания кросс-вендорных решений.

В последние годы особую популярность привлекает открытая архитектура набора команд RISC-V~\cite{waterman2014risc}. В отличие от проприетарных архитектур x86 и ARM, RISC-V является полностью открытой спецификацией, доступной для использования, модификации и расширения без лицензионных отчислений. Открытость RISC-V стимулирует появление новых процессоров и ускорителей, в том числе для встраиваемых систем и одноплатных компьютеров. Экосистема RISC-V активно развивается: появляются производительные многоядерные процессоры, поддержка векторных расширений и реализации OpenCL, что открывает возможности для высокопроизводительных вычислений на этой архитектуре.

Однако эффективность существующих библиотек линейной алгебры на платформах RISC-V остаётся малоисследованной областью. Различия в архитектуре процессоров, особенности реализации инструкций, размеры кэш-памяти и характеристики подсистемы памяти могут существенно влиять на производительность различных алгоритмических подходов к реализации GEMM. Понимание этих особенностей важно для разработки эффективных вычислительных приложений на RISC-V платформах.

Одной из наиболее зрелых и оптимизированных реализаций BLAS операций для OpenCL является библиотека CLBlast~\cite{clblast}, которая включает встроенную систему автоматического тюнинга параметров под конкретное оборудование. Сравнительный анализ производительности учебной библиотеки MyGEMM и промышленной библиотеки CLBlast на платформах RISC-V позволяет оценить как эффективность различных подходов к оптимизации, так и степень зрелости программной экосистемы для этой архитектуры.