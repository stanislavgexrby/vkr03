% !TeX spellcheck = ru_RU
% !TEX root = vkr.tex

\section*{Введение}
\thispagestyle{withCompileDate}

Разработка языков программирования, формальных спецификаций и систем обработки данных неразрывно связана с созданием и анализом их синтаксиса. Для описания синтаксиса широко используются формальные грамматики, такие как RBNF (Regular-Backus–Naur Form), которые, будучи мощным инструментом для специалистов, остаются машинно-ориентированными и плохо воспринимаются человеком в своем текстовом виде. Визуализация синтаксических правил в форме диаграмм, подобных диаграммам Вирта, значительно упрощает их понимание, анализ и использование, что особенно важно в образовательном процессе, при проектировании новых языков и при интеграции или модифицировании существующих систем.

В современном мире, характеризующемся ростом сложности программных систем и разнообразия предметно-ориентированных языков (DSL), инструменты, облегчающие взаимодействие человека с формальными спецификациями, приобретают ключевую важность. Они находят применение в областях, выходящих далеко за рамки классического проектирования компиляторов: от проектирования протоколов связи и форматов данных до конфигурирования бизнес-процессов и создания систем сценариев в игровых движках и корпоративных приложениях.

Исторически многие из таких специализированных инструментов, включая систему SynGT для построения и преобразования управляющих граф-схем по RBNF-грамматикам, были разработаны на языках вроде Pascal, обладающих ограниченной экосистемой и слабой интеграцией с современными программными системами. Это создаёт технологический разрыв: ценная функциональность остается в устаревших средах выполнения, что затрудняет её использование, развитие и интеграцию в современные процессы разработки, основанные на CI/CD, контейнеризации и веб-технологиях.

Таким образом, работа по эквивалентному портированию инструментальной системы SynGT представляет собой не просто механический перевод кода, а актуальную научно-практическую задачу по сохранению и актуализации методологии визуализации и преобразования синтаксических диаграмм.