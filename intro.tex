% !TeX spellcheck = ru_RU
% !TEX root = vkr.tex

\section*{Введение}
\thispagestyle{withCompileDate}

Операции линейной алгебры используются во многих областях: машинное обучение, компьютерное зрение, научные расчёты, финансовое моделирование. Среди них операция умножения матриц общего вида (GEMM, General Matrix Multiply) является наиболее зависимой от производительности, поскольку её эффективная реализация определяет скорость работы алгоритмов свёрточных нейронных сетей, итерационных решателей систем линейных уравнений и многих других методов.

Для ускорения вычислений используются графические процессоры (GPU) с массивно-параллельной архитектурой. Разработка ПО для GPU осложняется разнообразием платформ: производители (NVIDIA, AMD, Intel) предлагают разные архитектуры и программные интерфейсы. Наиболее распространённый API для GPU вычислений --- CUDA~\cite{nickolls2008cuda} --- является проприетарной технологией NVIDIA и не поддерживается на устройствах других производителей.

Альтернативой проприетарным решениям служит открытый стандарт OpenCL~\cite{opencl_spec} (Open Computing Language) --- кросс-платформенный фреймворк для параллельного программирования, позволяющий разрабатывать приложения, исполняемые на различных вычислительных устройствах независимо от производителя. OpenCL обеспечивает переносимость кода и абстрагирование от специфики аппаратной архитектуры, что делает его резонным выбором для создания кросс-вендорных решений.

В последние годы особую популярность привлекает открытая архитектура набора команд RISC-V~\cite{waterman2014risc}. В отличие от проприетарных архитектур x86 и ARM, RISC-V является полностью открытой спецификацией, доступной для использования, модификации и расширения без лицензионных отчислений. Открытость RISC-V стимулирует появление новых процессоров и ускорителей, в том числе для встраиваемых систем и одноплатных компьютеров. Экосистема RISC-V активно развивается: появляются производительные многоядерные процессоры, поддержка векторных расширений и реализации OpenCL, что открывает возможности для высокопроизводительных вычислений на этой архитектуре.

Эффективность библиотек линейной алгебры на платформах RISC-V пока мало изучена. Архитектура процессоров, реализация инструкций, размеры кэш-памяти и, что особенно важно, особенности работы с графическими ускорителями, в частности, GPU от Imagination Technologies, которые широко используются в данной экосистеме, существенно влияют на производительность разных подходов к реализации GEMM. Понимание этих особенностей важно для разработки эффективных приложений для архитектуры RISC-V.

CLBlast~\cite{clblast} — одна из наиболее развитых реализаций BLAS для OpenCL, включающая систему автоматического тюнинга параметров. Сравнение учебной библиотеки MyGEMM и промышленной CLBlast на платформах RISC-V позволяет оценить эффективность различных подходов к оптимизации и состояние программной экосистемы для этой архитектуры.