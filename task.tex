% !TeX spellcheck = ru_RU
% !TEX root = vkr.tex

\section{Постановка задачи}
\label{sec:task}

Целью данной работы является экспериментальное исследование производительности различных реализаций алгоритма умножения матриц из библиотеки MyGEMM\footnote{Реализация библиотеки на github: \url{https://github.com/SparseLinearAlgebra/myGEMM} (Дата посещения: 11.02.2025)} на одноплатных компьютерах с архитектурой RISC-V. В рамках работы проводится сравнительный анализ эффективности семи ядер библиотеки на двух аппаратных платформах --- Banana Pi BPI-F3\footnote{Документация Banana Pi: \url{https://wiki.banana-pi.org/Banana_Pi_BPI-F3} (Дата посещения: 12.02.2025)} и StarFive VisionFive~2\footnote{Документация StarFive: \url{https://www.starfivetech.com/en/site/boards}(Дата посещения: 13.02.2025)}, выявляются особенности поведения различных оптимизаций на RISC-V архитектуре, а также определяются оптимальные конфигурации параметров для достижения максимальной производительности. Для достижения поставленной выше цели необходимо выполнить следующие задачи.
\begin{enumerate}
    \item Обновление системных пакетов и программного обеспечения на тестируемых платформах для обеспечения совместимости с библиотекой MyGEMM.
    \item Сборка и проверка базовой работоспособности библиотеки MyGEMM в целевой среде.
    \item Провести серию тестов, направленных на измерение производительности различных ядер библиотеки MyGEMM на выбранных аппаратных платформах.
    \item Выявить причины различий в эффективности работы ядер, включая особенности архитектуры процессоров, реализации инструкций и механизмов управления памятью.
    \item Провести сравнение полученных результатов и сформулировать рекомендации по выбору наиболее подходящих реализаций для различных аппаратных условий.
\end{enumerate}