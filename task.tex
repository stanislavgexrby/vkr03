% !TeX spellcheck = ru_RU
% !TEX root = vkr.tex

\section{Постановка задачи}
\label{sec:task}

Цель данной работы — портирование инструментальной системы SynGT с языка Pascal на современный язык C++ с сохранением и улучшением её функциональных возможностей. Портируемая система должна обеспечить построение и эквивалентные преобразования управляющих граф-схем по RBNF-грамматикам, а также предоставить пользователю интуитивно понятный графический интерфейс для визуализации и редактирования диаграмм.

Для достижения поставленной цели необходимо выполнить следующие задачи.

\begin{enumerate}
\item Провести анализ исходного кода системы SynGT на языке Pascal. Выявить ключевые модули, алгоритмы (в частности, алгоритм размещения объектов) и структуры данных, подлежащие портированию.
\item Разработать архитектуру портированной системы на C++. Спроектировать модульную структуру, обеспечивающую разделение логики разбора грамматик, алгоритмов визуализации и компонентов пользовательского интерфейса.
\item Реализовать ядро системы на C++ — библиотеку, выполняющую разбор RBNF-грамматик и построение соответствующих управляющих граф-схем.
\item Разработать набор тестов для верификации корректности работы портированной библиотеки.
\item Создать консольное приложение для библиотеки, позволяющее проводить её дополнительную проверку и использовать в качестве самостоятельного инструмента для пакетной обработки грамматик.
\item Реализовать приложение с графическим пользовательским интерфейсом, обеспечивающее визуальное построение, редактирование, масштабирование и сохранение синтаксических диаграмм.
\end{enumerate}

В результате выполнения работы будет создана современная, поддерживаемая и расширяемая инструментальная система для работы с RBNF-грамматиками.