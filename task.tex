% !TeX spellcheck = ru_RU
% !TEX root = vkr.tex

\section{Постановка задачи}
\label{sec:task}

Целью данной работы является экспериментальное исследование производительности различных реализаций алгоритма умножения матриц из библиотек MyGEMM~\cite{mygemm_repo} и CLBlast~\cite{clblast} на одноплатных компьютерах с архитектурой RISC-V. В данной работе приведен сравнительный анализ эффективности ядер библиотеки на двух аппаратных платформах --- Banana Pi BPI-F3~\cite{bananapi_f3_wiki} и StarFive VisionFive~2~\cite{visionfive2_wiki}, выявляются особенности поведения различных оптимизаций на RISC-V архитектуре, а также определяются оптимальные конфигурации параметров для достижения максимальной производительности. Также проводится исследование производительности промышленной библиотеки CLBlast с использованием встроенной системы автоматического тюнинга параметров под конкретное оборудование. Сравнительный анализ учебной библиотеки MyGEMM и промышленной библиотеки CLBlast позволяет оценить эффективность различных подходов к оптимизации на платформах RISC-V.

Для достижения поставленной выше цели необходимо выполнить следующие задачи.
\begin{enumerate}
    \item Обновление системных пакетов и программного обеспечения на тестируемых платформах для обеспечения совместимости с библиотеками MyGEMM и CLBlast.
    \item Провести серию тестов производительности различных ядер библиотеки MyGEMM на выбранных аппаратных платформах.
    \item Выполнить процедуру автоматического тюнинга библиотеки CLBlast для RISC-V платформ и провести измерения производительности.
    \item Провести сравнительный анализ производительности MyGEMM и CLBlast, выявить причины различий в эффективности работы, включая особенности архитектуры процессоров, реализации инструкций и механизмов управления памятью.
\end{enumerate}