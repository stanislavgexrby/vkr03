% !TeX spellcheck = ru_RU
% !TEX root = vkr.tex

\section{Эксперимент с библиотекой CLBlast}

\subsection{Установка и настройка}

Для проведения экспериментального исследования производительности CLBlast на RISC-V платформах была выполнена установка библиотеки из исходных кодов. На обеих тестируемых платформах (Banana Pi BPI-F3 и StarFive VisionFive 2) использовалась следующая процедура:

\begin{enumerate}
    \item Клонирование репозитория CLBlast:
    \begin{verbatim}
    git clone https://github.com/CNugteren/CLBlast.git
    cd CLBlast
    \end{verbatim}
    
    \item Сборка библиотеки с использованием CMake:
    \begin{verbatim}
    mkdir build && cd build
    cmake ..
    make -j$(nproc)
    sudo make install
    \end{verbatim}
\end{enumerate}

После успешной установки была проведена проверка корректности работы библиотеки с помощью встроенных тестовых программ.

\subsection{Процесс автоматического тюнинга}

Ключевым этапом подготовки библиотеки CLBlast к эксперименту явилась процедура автоматического тюнинга параметров под конкретные характеристики тестируемых платформ. Тюнинг выполнялся отдельно для каждой платформы с использованием утилиты \texttt{clblast\_tuner\_xgemm}, входящей в состав библиотеки.

Процесс тюнинга для операции GEMM занял:
\begin{itemize}
    \item На Banana Pi BPI-F3: [будет заполнено после эксперимента]
    \item На StarFive VisionFive 2: [будет заполнено после эксперимента]
\end{itemize}

Результатом тюнинга явились оптимизированные конфигурационные файлы, автоматически используемые библиотекой при последующих вызовах.

\subsection{Условия эксперимента}

Эксперименты с CLBlast проводились на тех же аппаратных платформах и в тех же условиях, что и тестирование MyGEMM (раздел~\ref{sec:mygemm_conditions}). Для обеспечения сопоставимости результатов использовались идентичные тестовые данные: квадратные матрицы размером $1024 \times 1024$ элементов типа \texttt{float}.

Для каждой платформы выполнялось по 100 прогонов операции умножения матриц с измерением времени выполнения. Использовалось среднее арифметическое значение для оценки производительности.

\subsection{Результаты измерений}

\textit{[Этот раздел будет заполнен после получения реальных данных с платформ]}

Предварительные ожидаемые результаты на основе архитектурных особенностей:
\begin{itemize}
    \item Banana Pi BPI-F3: время выполнения 0.15--0.25 сек
    \item StarFive VisionFive 2: время выполнения 0.25--0.40 сек  
    \item Intel Iris Xe: время выполнения 0.003--0.005 сек
\end{itemize}

\subsection{Сравнительный анализ}

\subsubsection{Абсолютная производительность}

\textit{[Будет заполнено после эксперимента]}

Сравнение времени выполнения операции умножения матриц между лучшим ядром MyGEMM (ядро 11) и оптимизированной CLBlast.

\subsubsection{Эффективность на RISC-V}

\textit{[Будет заполнено после эксперимента]}

Анализ того, насколько эффективно промышленная библиотека с автотюнингом работает на платформах RISC-V по сравнению с ручными оптимизациями.

\subsubsection{Стабильность и предсказуемость}

\textit{[Будет заполнено после эксперимента]}

Оценка стабильности времени выполнения и влияния автотюнинга на предсказуемость производительности.