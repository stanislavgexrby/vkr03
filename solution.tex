% !TeX spellcheck = ru_RU
% !TEX root = vkr.tex

\section{Первый запуск}

При первом запуске библиотеки MyGEMM~\cite{mygemm_repo} возникла ошибка выполнения OpenCL ядер. Сообщение об ошибке указывало на проблему с созданием рабочих групп (work-groups), однако точная причина не была очевидна.

Для выяснения причины был проанализирован исходный код библиотеки. Изучение реализации различных ядер показало, что ключевыми параметрами, определяющими размер рабочих групп, являются константы TS (Tile Size), TSM (Tile Size M) и TSN (Tile Size N). Согласно документации MyGEMM~\cite{nugteren2018mygemm}, эти параметры определяют размерность блоков данных, обрабатываемых одной рабочей группой. Для ядер 1--3 размер рабочей группы составляет $\text{TS} \times \text{TS}$ потоков, для ядер 4--10 размер определяется как $\frac{\text{TSM}}{\text{WPTM}} \times \frac{\text{TSN}}{\text{WPTN}}$ потоков.

Значения по умолчанию в библиотеке установлены для дискретных GPU NVIDIA. Для ядер 1--3 используется TS = 32, что создаёт рабочие группы размером $32 \times 32 = 1024$ потока. Для ядер 6--10 используются TSM = 128, TSN = 128, WPTM = 8, WPTN = 8, что даёт размер группы $\frac{128}{8} \times \frac{128}{8} = 16 \times 16 = 256$ потоков.

Для определения аппаратных ограничений платформ была использована утилита \texttt{clinfo}~\cite{clinfo}. Анализ вывода утилиты на обеих платформах показал, что максимальный размер рабочей группы (параметр \texttt{CL\_DEVICE\_MAX\_WORK\_GROUP\_SIZE}) составляет 512 потоков. Это существенно отличается от типичных значений для дискретных GPU (1024--2048 потоков), что объясняет ошибку при запуске с параметрами по умолчанию: размер группы 1024 потока превышает аппаратное ограничение.

С учётом выявленного ограничения были модифицированы параметры запуска для всех ядер библиотеки. Для ядер 1--3 параметр TS был установлен в значения 8 и 16, что даёт размеры рабочих групп 64 и 256 потоков соответственно. Для ядер 4--10 были подобраны комбинации параметров TSM и TSN из множества \{32, 64, 128\} с соответствующими значениями WPTM и WPTN, обеспечивающие соблюдение ограничения на размер рабочей группы. После внесения изменений библиотека успешно запустилась на обеих платформах, что позволило перейти к исследованию производительности различных ядер.