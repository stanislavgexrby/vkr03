% !TeX spellcheck = ru_RU
% !TEX root = vkr.tex

\section{Обзор}
\label{sec:relatedworks}

\subsection{Контекст исследования}

Операция умножения матриц общего вида (GEMM) является критически важным компонентом многих вычислительных приложений. Для её эффективной реализации на GPU используются различные программные интерфейсы, среди которых выделяются два основных подхода: проприетарная технология CUDA от NVIDIA и открытый стандарт OpenCL.

OpenCL обеспечивает переносимость кода между устройствами различных производителей, что делает его привлекательным выбором для разработки кросс-платформенных решений. Однако эффективность реализаций GEMM на OpenCL сильно зависит от аппаратной платформы и применённых техник оптимизации.

В контексте развития архитектуры RISC-V появляются новые вычислительные платформы с поддержкой OpenCL, производительность которых остаётся малоисследованной. Понимание того, какие оптимизации эффективны на таких платформах, важно для будущего развития эффективных и производительных решений.

Библиотека MyGEMM предоставляет набор последовательно оптимизируемых реализаций GEMM с открытым исходным кодом, что делает её удобным инструментом для экспериментального исследования влияния различных оптимизаций на производительность конкретного оборудования.

\subsection{Библиотека MyGEMM}

Работа основана на открытом туториале MyGEMM~\cite{nugteren2018mygemm}, который демонстрирует пошаговую оптимизацию умножения матриц на GPU. Туториал содержит описания 11 последовательных версий ядра (kernel), каждое из которых добавляет определённые оптимизации. Ниже приведено краткое описание каждого ядра:

Ядро 1 (myGEMM1) --- наивная реализация, где каждый поток вычисляет один элемент результирующей матрицы. Служит базовым примером для сравнения эффективности оптимизаций. Производительность крайне низкая из-за неэффективного использования памяти и отсутствия переиспользования данных.

Ядро 2 (myGEMM2) --- блочный алгоритм с использованием локальной памяти (tiled algorithm)~\cite{goto2008anatomy}. Вводится параметр $\text{TS}$ (tile size), определяющий размер блока матрицы. Каждая рабочая группа загружает блок размера $\text{TS} \times \text{TS}$ в локальную память, что позволяет многократно использовать данные и снижает количество обращений к глобальной памяти. Барьерная синхронизация обеспечивает корректность работы с общей локальной памятью.

Ядро 3 (myGEMM3) --- оптимизация использования кэша путём подсчёта нескольких элементов одним потоком. Вводятся параметры $\text{WPT}$ (work-per-thread) --- число элементов, вычисляемых одним потоком, а также $\text{RTS} = \text{TS}/\text{WPT}$ --- сокращённый размер рабочей группы. Это позволяет лучше использовать данные, загруженные в регистры, и повысить арифметическую интенсивность.

Ядро 4 (myGEMM4) --- двумерное блокирование в регистрах (2D register blocking). Вместо одного параметра $\text{WPT}$ вводятся два независимых параметра: $\text{WPTM}$ и $\text{WPTN}$ --- количество элементов, вычисляемых одним потоком по каждому измерению матрицы. Также вводятся параметры $\text{TSM}$ и $\text{TSN}$ --- размеры блоков по соответствующим измерениям. Это позволяет более гибко настроить алгоритм под архитектурные особенности процессора.

Ядро 5 (myGEMM5) --- транспонирование локальной памяти для улучшения паттерна доступа. Матрица $A$ загружается в локальную память с транспонированием, что обеспечивает более эффективные последовательные обращения к памяти при выполнении вычислений.

Ядро 6 (myGEMM6) --- векторизация операций загрузки/записи. Вводится параметр $\text{VW}$ (vector width) --- ширина вектора для загрузки данных. Использование векторных типов данных OpenCL позволяет за одну операцию загружать/записывать несколько элементов, что увеличивает пропускную способность памяти.

Ядро 7 (myGEMM7) --- выравнивание глобальной памяти. Обеспечивается правильное выравнивание (alignment) доступа, что необходимо для эффективной работы векторных операций загрузки и записи.

Ядро 8 (myGEMM8) --- двумерная регистровая блокировка (2D register blocking). Развивает идею ядра 6, добавляя более сложную организацию вычислений, при которой каждый поток обрабатывает прямоугольный блок элементов результирующей матрицы, кэшируя промежуточные данные в регистрах процессора.

Ядро 9 (myGEMM9) --- добавление программной предвыборки данных (software pre-fetching). Использует техники асинхронной загрузки данных для сокрытия латентности обращений к памяти за счёт перекрытия загрузки следующего блока данных с вычислениями над текущим.

Ядро 10 (myGEMM10) --- поддержка произвольных размеров матриц. Добавляет обработку граничных случаев (edge cases), когда размеры матриц не кратны размерам блоков, обеспечивая корректную работу для матриц любых размеров.

Ядро 11 (myGEMM11) --- реализация подхода clBLAS~\cite{clblas2015}. В отличие от всех предыдущих ядер, это ядро \textbf{не использует локальную память}, полностью полагаясь на регистровую блокировку и векторные типы данных. Основные особенности:

\begin{itemize}
    \item Фиксированный малый размер рабочей группы (8×8 потоков);
    \item Использование векторных типов данных OpenCL (float8 для матрицы $A$, float4 для матрицы $B$);
    \item 2D регистровая блокировка с параметрами RX, RY, RK, определяющими размеры блока, обрабатываемого одним потоком;
    \item Прямая загрузка векторов данных из глобальной памяти в регистры процессора без промежуточного использования локальной памяти;
    \item Выполнение множественных операций умножения-сложения (FMA) над данными в регистрах перед загрузкой новой порции данных.
\end{itemize}

Каждое последующее ядро в туториале демонстрирует улучшение производительности на дискретных GPU NVIDIA. Однако эффективность этих оптимизаций на RISC-V платформах с интегрированными GPU может существенно отличаться, что и является предметом данного исследования.

\subsection{Конфигурационные параметры ядер}

Различные ядра MyGEMM используют различные наборы параметров для настройки производительности. Ниже приведено описание основных параметров оптимизации.

\subsubsection{Параметры базовых ядер (1--3)}

TS (Tile Size) --- размер квадратного блока матрицы, загружаемого в локальную память рабочей группой. Определяет размер рабочей группы ($\text{TS} \times \text{TS}$ потоков) и объём используемой локальной памяти ($2 \times \text{TS}^2$ элементов типа float). Большие значения увеличивают переиспользование данных, но требуют больше локальной памяти и могут снизить заполнение (occupancy).

WPT (Work Per Thread) --- количество элементов результирующей матрицы, вычисляемых одним потоком (используется в ядре 3). При увеличении WPT размер рабочей группы сокращается до $\text{RTS} = \text{TS}/\text{WPT}$ потоков на измерение, что позволяет лучше использовать регистры и повысить арифметическую интенсивность.

\subsubsection{Параметры продвинутых ядер (4--10)}

TSM и TSN --- размеры блоков локальной памяти по измерениям $M$ и $N$ соответственно. Обобщают параметр TS для асимметричных блоков, позволяя независимо настраивать размеры по каждому измерению. Размер рабочей группы определяется как $\frac{\text{TSM}}{\text{WPTM}} \times \frac{\text{TSN}}{\text{WPTN}}$ потоков.

WPTM и WPTN --- количество элементов результирующей матрицы, вычисляемых одним потоком по измерениям $M$ и $N$ соответственно. Двумерное обобщение параметра WPT. Каждый поток вычисляет прямоугольный блок размером $\text{WPTM} \times \text{WPTN}$ элементов.

VW (Vector Width) --- ширина вектора для операций загрузки и записи данных (используется в ядрах 6--10). Определяет количество элементов, загружаемых/записываемых одной векторной операцией. Типичные значения: 1, 2, 4, 8, 16. Требует, чтобы TSM, TSN, WPTM, WPTN были кратны VW. Увеличение VW повышает пропускную способность памяти, но может снизить гибкость конфигурации.

\subsubsection{Параметры ядра 11 (clBLAS-подход)}

Ядро 11 использует принципиально иной набор параметров, не связанный с локальной памятью:

RX и RY --- размеры 2D регистрового блока, обрабатываемого одним потоком. RX определяет количество элементов по измерению, соответствующему матрице $A$, RY --- по измерению матрицы $B$. В оригинальной реализации clBLAS используются значения RX = 8, RY = 4, соответствующие векторным типам float8 и float4.

RK --- размер блока по измерению $K$ (глубина матриц при умножении). Определяет количество итераций внутреннего цикла накопления до загрузки новой порции данных. В стандартной конфигурации RK = RY = 4, что позволяет эффективно переиспользовать данные матрицы $B$.

Размер рабочей группы в ядре 11 фиксирован и не зависит от параметров: 8×8 = 64 потока в оригинальной реализации clBLAS, либо настраиваемый параметр в модифицированных версиях.

\subsection{Библиотека CLBlast}

CLBlast~\cite{clblast} --- современная высокопроизводительная библиотека базовых операций линейной алгебры (BLAS) для OpenCL, разработанная Cedric Nugteren. В отличие от учебной библиотеки MyGEMM, CLBlast — это полнофункциональная библиотека, предназначенная для практического применения.

\subsubsection{Основные характеристики}

CLBlast реализует практически полный набор операций BLAS уровней 1, 2 и 3, включая операцию умножения матриц общего вида (GEMM). Библиотека спроектирована с акцентом на переносимость и производительность, обеспечивая эффективную работу на широком спектре устройств с поддержкой OpenCL --- от встроенных GPU до дискретных графических ускорителей профессионального уровня.

Ключевые особенности CLBlast:

\begin{itemize}
    \item \textbf{Полная реализация BLAS}: поддержка операций уровней 1 (векторные операции), 2 (матрично-векторные операции) и 3 (матрично-матричные операции);
    \item \textbf{Кросс-платформенность}: работа на GPU различных производителей (NVIDIA, AMD, Intel, ARM Mali, Imagination PowerVR) через единый интерфейс OpenCL;
    \item \textbf{Автоматический тюнинг}: встроенная система автоматической оптимизации параметров под конкретную аппаратную платформу;
    \item \textbf{Открытый исходный код}: распространяется под лицензией Apache 2.0, что обеспечивает прозрачность реализации и возможность модификации.
\end{itemize}

\subsubsection{Механизм автоматического тюнинга}

CLBlast отличается от MyGEMM встроенной системой автоматической настройки производительности. Библиотека поставляется с базой предварительно оптимизированных конфигураций для наиболее распространённых GPU, однако для достижения максимальной производительности на новых или нестандартных платформах (таких как RISC-V SBC) необходимо провести процедуру тюнинга.

Процесс тюнинга представляет собой систематический перебор значений конфигурационных параметров с измерением производительности на реальном оборудовании. Для операции GEMM автотюнер варьирует следующие основные параметры:

\begin{itemize}
    \item \textbf{MWG, NWG, KWG} --- размеры рабочих групп по измерениям M, N и K;
    \item \textbf{MDIMC, NDIMC} --- количество потоков на измерение внутри рабочей группы;
    \item \textbf{MDIMA, NDIMB} --- параметры загрузки данных из глобальной памяти;
    \item \textbf{KWI} --- количество итераций внутреннего цикла;
    \item \textbf{VWM, VWN} --- ширина векторизации операций по измерениям M и N;
    \item \textbf{STRM, STRN} --- параметры использования локальной памяти;
    \item \textbf{SA, SB} --- стратегии кэширования матриц A и B.
\end{itemize}

Автотюнер использует алгоритмы поиска в пространстве параметров для выявления оптимальных конфигураций, минимизируя время выполнения операции при соблюдении аппаратных ограничений (размер локальной памяти, максимальный размер рабочей группы, количество регистров).

\subsubsection{Отличия от MyGEMM}

Основные различия CLBlast и MyGEMM приведены в таблице~\ref{tab:mygemm_vs_clblast}.

\begin{table}[h]
\centering
\caption{Сравнение библиотек MyGEMM и CLBlast}
\label{tab:mygemm_vs_clblast}
\begin{tabular}{p{0.45\textwidth}p{0.45\textwidth}}
\hline
\textbf{MyGEMM} & \textbf{CLBlast} \\
\hline
Учебная библиотека, демонстрирующая пошаговую оптимизацию & Промышленная библиотека для производственного использования \\
\hline
11 отдельных ядер с нарастающей сложностью оптимизаций & Единое универсальное ядро с адаптивными параметрами \\
\hline
Ручная настройка параметров для каждой платформы & Автоматический поиск оптимальных параметров \\
\hline
Фокус на понимании принципов оптимизации & Фокус на максимальной производительности \\
\hline
\end{tabular}
\end{table}

Если MyGEMM демонстрирует пошаговые оптимизации для обучения, то CLBlast предоставляет готовое решение для практического применения в задачах линейной алгебры.