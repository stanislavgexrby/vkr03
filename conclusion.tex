% !TeX spellcheck = ru_RU
% !TEX root = vkr.tex

\section*{Заключение}

В рамках научно-исследовательской работы выполнено портирование инструментальной системы SynGT с языка Pascal на современный язык C++. Все поставленные задачи выполнены.

Проведён детальный анализ исходного, в результате которого были выявлены ключевые алгоритмы (включая рекурсивный алгоритм размещения объектов с семью специализированными функциями) и структуры данных, необходимые для воспроизведения функциональности. Разработана модульная архитектура портированной системы, обеспечивающая четкое разделение компонентов разбора грамматик, алгоритмов визуализации и пользовательского интерфейса.

Реализовано ядро системы в виде библиотеки на C++, обеспечивающее полную функциональность разбора RBNF-грамматик и построения управляющих граф-схем. Функциональная эквивалентность с оригинальной реализацией подтверждена разработанным комплексом тестов. Создано консольное приложение для пакетной обработки грамматик и дополнительной верификации корректности работы системы.

Завершающим этапом стала разработка графического приложения с пользовательским интерфейсом, предоставляющего возможности интерактивного построения, редактирования и визуализации синтаксических диаграмм. Приложение поддерживает функции оригинальной системы, включая работу со вспомогательными понятиями, механизмы выделения и перемещения объектов, а также сохранение результатов.

В результате создана современная инструментальная система для эквивалентных преобразований синтаксических диаграмм, обладающая улучшенной архитектурой, кроссплатформенностью и потенциалом для дальнейшего развития и интеграции в современные инструментальные цепочки разработки.

\subsection*{Дальнейшие направления}

Перспективы развития работы связаны со следующими направлениями.

\begin{itemize}
\item Расширение поддерживаемого подмножества RBNF-грамматик и добавление новых конструкций для описания синтаксиса.
\item Оптимизация алгоритмов визуализации для работы с грамматиками большого объёма и сложности.
\item Разработка дополнительных форматов экспорта диаграмм (SVG, PDF) и интеграция с системами непрерывной интеграции.
\item Создание веб-версии системы для обеспечения доступности через браузер.
\end{itemize}