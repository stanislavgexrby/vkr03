% !TeX spellcheck = ru_RU
% !TEX root = vkr.tex

\section*{Заключение}

В рамках данной работы было проведено комплексное экспериментальное исследование производительности библиотек MyGEMM и CLBlast на одноплатных компьютерах с архитектурой RISC-V. Все поставленные задачи были успешно выполнены.

\textbf{Основные результаты работы:}

\begin{enumerate}
    \item \textbf{Адаптация MyGEMM для RISC-V платформ.} Библиотека MyGEMM была успешно адаптирована для работы на одноплатных компьютерах Banana Pi BPI-F3 и StarFive VisionFive 2. Выявлено критическое аппаратное ограничение в 32 потока на рабочую группу, что потребовало модификации параметров запуска всех 11 ядер библиотеки. Разработаны конфигурации параметров (TS, TSM, TSN, WPTM, WPTN), обеспечивающие корректное выполнение ядер с учётом данного ограничения.
    
    \item \textbf{Экспериментальное тестирование MyGEMM.} Проведено комплексное экспериментальное исследование всех 11 ядер библиотеки MyGEMM на двух RISC-V платформах и референсной системе Intel Iris Xe. Для каждой конфигурации параметров выполнено по 100 прогонов операции умножения матриц размером $1024 \times 1024$ элементов. Разработаны скрипты автоматизации тестирования, обеспечивающие воспроизводимость результатов.
    
    \item \textbf{Выявление особенностей производительности MyGEMM на RISC-V.} Установлено, что ядро 11 (clBLAS-подход с регистровой блокировкой без использования локальной памяти) демонстрирует наилучшую производительность на RISC-V платформах, обеспечивая ускорение в 16--18 раз относительно наивной реализации и время выполнения 0.49 сек на StarFive VisionFive 2 для матриц 1024×1024 (4.38 GFLOPS). Классические оптимизации с использованием локальной памяти (ядра 2--5) показывают ускорение в 10--25 раз, что подтверждает их эффективность на встроенных графических ядрах RISC-V процессоров.
    
    \item \textbf{Обнаружение критической проблемы векторизации в MyGEMM.} Выявлена серьёзная проблема производительности векторизованных ядер (6--10): они выполняются в 600--1000 раз медленнее на RISC-V платформах по сравнению с Intel Iris Xe. Сформулирована обоснованная гипотеза о причинах деградации: неоптимизированная компиляция векторных операций загрузки/записи (\texttt{vload}/\texttt{vstore}), возможная эмуляция на CPU и проблемы с выравниванием данных в драйверах OpenCL для архитектуры PowerVR BXE.
    
    \item \textbf{Тюнинг и тестирование CLBlast.} Успешно выполнена процедура автоматического тюнинга библиотеки CLBlast для платформ StarFive VisionFive 2 и Banana Pi BPI-F3. Процесс тюнинга включал оптимизацию всех ядер, задействованных в операции GEMM, с учётом специфических аппаратных ограничений RISC-V архитектуры. Проведено тестирование производительности как с параметрами по умолчанию, так и после автоматической оптимизации на диапазоне размеров матриц от 512×512 до 7680×7680 элементов.
    
    \item \textbf{Неожиданный эффект автоматического тюнинга.} Обнаружено контринтуитивное поведение автотюнера CLBlast на RISC-V платформах: процедура автоматической оптимизации привела к ухудшению производительности на 15.6\% для матриц 1024×1024 (с 0.349 сек до 0.403 сек) и на 75.6\% для малых матриц 512×512 (с 0.048 сек до 0.085 сек). При этом на больших матрицах (7680×7680) различие минимально (0.13\%). Сформулирована гипотеза о причинах: автотюнер оптимизирован для типичных GPU с большими размерами рабочих групп (256--1024 потока) и не адаптирован для жёстких ограничений RISC-V платформ (32 потока).

    \item \textbf{Сравнительный анализ библиотек.} Установлено, что CLBlast с параметрами по умолчанию демонстрирует наилучшую производительность среди всех тестируемых реализаций --- 0.349 сек (6.16 GFLOPS) на матрицах 1024×1024, что на 40\% быстрее лучшего ядра MyGEMM (0.49 сек, 4.38 GFLOPS). Это подтверждает зрелость промышленной библиотеки и высокое качество её базовых параметров для архитектуры PowerVR BXE.
    
    Однако после тюнинга CLBlast замедляется до 0.403 сек (5.32 GFLOPS), что всё ещё на 18\% быстрее MyGEMM, но существенно хуже исходной конфигурации. Это критически важное наблюдение для практического применения автотюнинга на нестандартных архитектурах.
\end{enumerate}

Исходный код с модификациями параметров запуска и скрипты автоматизации тестирования доступны в форке репозитория MyGEMM~\cite{mygemm_repo_test}.

\subsection*{Научный вклад и практическая значимость}

Работа вносит существенный вклад в понимание особенностей высокопроизводительных вычислений на развивающейся экосистеме RISC-V:

\begin{enumerate}
    \item \textbf{Первое систематическое исследование} производительности библиотек линейной алгебры на RISC-V одноплатных компьютерах с GPU PowerVR BXE, заполняющее существенный пробел в литературе.
    
    \item \textbf{Выявление ключевого ограничения} RISC-V OpenCL реализаций (максимум 32 потока на рабочую группу) и разработка методологии адаптации существующих библиотек к этому ограничению.
    
    \item \textbf{Обнаружение контринтуитивного поведения} автоматического тюнинга на нестандартных архитектурах, что имеет важное значение для разработчиков библиотек и пользователей HPC систем на базе RISC-V.
\end{enumerate}

Результаты работы представляют практическую ценность для:
\begin{itemize}
    \item Разработчиков приложений машинного обучения и научных вычислений на RISC-V платформах
    \item Производителей RISC-V процессоров и разработчиков драйверов OpenCL
    \item Сообщества разработчиков библиотек линейной алгебры (CLBlast, OpenBLAS и др.)
    \item Образовательных программ по параллельному программированию и оптимизации вычислений
\end{itemize}

\subsection*{Дальнейшие направления исследований}

В дальнейшем работа над исследованием производительности библиотек линейной алгебры на RISC-V платформах может включать:

\begin{itemize}
    \item \textbf{Профилирование выполнения} векторизованных ядер с использованием специализированных инструментов анализа производительности OpenCL для подтверждения гипотезы о причинах деградации и выявления узких мест в драйверах.
    
    \item \textbf{Доработка автотюнера CLBlast} для корректной работы с архитектурами, имеющими малые размеры рабочих групп, включая расширение пространства поиска параметров и модификацию эвристик оптимизации.
    
    \item \textbf{Тестирование на новых RISC-V платформах} с улучшенными драйверами OpenCL, увеличенными аппаратными ограничениями на размер рабочих групп и поддержкой векторного расширения RVV для оценки прогресса экосистемы.
    
    \item \textbf{Разработка специализированных оптимизаций} ядер GEMM, учитывающих архитектурные особенности RISC-V процессоров и характеристики их графических ядер PowerVR, с потенциалом превзойти существующие реализации.
    
    \item \textbf{Расширение исследования на другие операции BLAS} (уровней 1 и 2), доступные в CLBlast, для комплексной оценки производительности библиотеки на RISC-V и выявления паттернов оптимизации.
    
    \item \textbf{Сравнительный анализ с OpenBLAS} и другими библиотеками линейной алгебры, поддерживающими RISC-V архитектуру, для определения лучших практик реализации BLAS операций на этой платформе.
    
    \item \textbf{Исследование масштабируемости} на многопроцессорных RISC-V системах и оценка эффективности MPI/OpenMP параллелизации в сочетании с OpenCL ускорением.
    
    \item \textbf{Разработка рекомендаций} для производителей драйверов OpenCL по улучшению поддержки векторных операций, оптимизации компиляции ядер и реализации эффективных паттернов доступа к памяти на RISC-V архитектуре.
\end{itemize}

Продолжение данного исследования позволит способствовать развитию высокопроизводительных вычислений на открытой архитектуре RISC-V и повышению конкурентоспособности этой платформы в области научных и инженерных расчётов.