% !TeX spellcheck = ru_RU
% !TEX root = vkr.tex

\section*{Заключение}

Результаты работы:

\begin{itemize}
    \item Библиотека MyGEMM была успешно адаптирована для работы на RISC-V платформах с учётом аппаратного ограничения в 32 потока на рабочую группу.
    
    \item Проведено экспериментальное исследование производительности 11 ядер библиотеки MyGEMM на двух RISC-V платформах (Banana Pi BPI-F3 и StarFive VisionFive 2) и одной референсной платформе Intel Iris Xe с тестированием различных конфигураций параметров.
    
    \item Выявлено, что ядро 11 (clBLAS-подход с регистровой блокировкой без использования локальной памяти) демонстрирует наилучшую производительность на RISC-V платформах, обеспечивая ускорение в 16--18 раз относительно наивной реализации.
    
    \item Обнаружена критическая проблема с производительностью векторизованных ядер (6--10) на RISC-V платформах: они выполняются в 600--1000 раз медленнее, чем на Intel Iris Xe, что указывает на недостаточную оптимизацию драйверов OpenCL для векторных операций на архитектуре RISC-V.
    
    \item Установлено, что классические оптимизации с использованием локальной памяти и блочной обработки (ядра 2--5) эффективно работают на RISC-V платформах, обеспечивая ускорение в 10--25 раз относительно наивной реализации.
    
    \item Сформулированы практические рекомендации по выбору оптимизаций для RISC-V платформ: предпочтение регистровой блокировке без локальной памяти, избегание векторизации до улучшения драйверов.
\end{itemize}

Исходный код с модификациями параметров запуска и скрипты для автоматизации тестирования доступны в форке репозитория MyGEMM~\cite{mygemm_repo_test}.

\subsection*{Дальнейшее направление исследований}

В дальнейшем работа над исследованием производительности библиотек линейной алгебры на RISC-V платформах может включать:

\begin{itemize}
    \item Исследование причин низкой производительности векторизованных операций на RISC-V платформах путём профилирования выполнения ядер с использованием специализированных инструментов анализа производительности OpenCL.
    
    \item Тестирование библиотеки MyGEMM на более новых RISC-V платформах с улучшенными драйверами OpenCL и большими аппаратными ограничениями на размер рабочих групп.
    
    \item Разработка специализированных оптимизаций ядер MyGEMM, учитывающих архитектурные особенности RISC-V процессоров и ограничения их графических ядер.
    
    \item Сравнительный анализ производительности MyGEMM с другими библиотеками линейной алгебры, поддерживающими RISC-V архитектуру.
    
    \item Исследование возможности адаптации других вычислительных библиотек для эффективной работы на RISC-V платформах на основе полученных результатов.
\end{itemize}